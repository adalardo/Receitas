\section{Pão de queijo}

\subsection*{Ingredientes}
\begin{itemize}
\item 500 g de polvilho azedo
\item 500 g de polvilho doce
\item 500 g de queijo mineiro meia-cura ralado
\item 500 ml de leite
\item 250 ml de água
\item 200 ml de óleo
\item 15 a 30 g de sal
\item 6 ovos batidos
\end{itemize}

\subsection*{Preparo}

\begin{enumerate}
\item Misture os dois tipos de polvilho em uma bacia.
\item Misture os líquidos e o sal em uma panela grande e leve ao fogo.
\item Quando os líquidos começarem a ferver baixe o fogo para não transbordar e deixe ferver mais 2-5 minutos. Em seguida jogue sobre o polvilho e misture o melhor possível. Em seguida trabalhe a massa com as mãos até que fique homogênea e elástica.
\item Acrescente o queijo e trabalhe até que esteja bem incorporado à massa.
\item Acrescente os ovos e trabalhe mais, até incoporar. A massa ficará pegajosa.
\item Aqueça o forno a 200 graus
\item Unte as mãos com óleo e faça bolinhas com a massa. Coloque-as em uma forma com espaçamento para crescer em trono de 25\%. Não precisa untar a forma.
\item Leve para assar até que esteja corado.
\end{enumerate}

\subsection*{Macetes}
\begin{itemize}
\item Polvilho de qualidade é essencial. Dos industrializados use Amafil ou Marinez.
\item Queijo então nem se fala! Tem que ser mineiro. E bão, uai.
\item Ovos capiras também valorizam o pão de queijo, que ficam mais amarelos.
\item A quantidade do sal é indicativa, e depende muito do queijo (prove-o) e do gosto. Experimente algumas vezes que você chegará ao seu ponto.
\item A massa é difícil de trabalhar no começo, porque está quente e gruda muito. Use luvas de silicone descartáveis.
\item A massa ou os pães enrolados podem ser congelados. Asse em forno um pouco mais quente no início, congelados, ou não.
\end{itemize}

\subsubsection*{Variações}
\begin{itemize}
\item Para um pão de queijo temperado misture bem dois dentes de alho amassados à massa, antes de enrolar.
\item \textbf{Uaiffle}: asse um pouco de massa na máquina de waffle. É rápido e pode ser recheado como um sanduíche.
\item \textbf{Triângulo mineiro}: asse um pouco de massa em saduicheira elétrica. Idem!
\end{itemize}

