\section{Pudim de leite condensado}
\subsection{Ingredientes}
\textbf{Pudim}
\begin{itemize}
\item 2 ovos
\item 6 gemas
\item 1 lata de leite condensado (395 g)
\item 1,5 lata de leite integral (380 g)
\end{itemize}
\textbf{Calda}
\begin{itemize}
\item 4 xícaras de açúcar cristal
\item 2 xícaras de água
\item 1 colher de sopa de vinagre
\end{itemize}
\subsection{Preparo}
\textbf{Calda}
\begin{enumerate}
\item Aqueça o açúcar em panela grande e mexa enquanto derrete. Continue mexendo até que fique cor de caramelo claro.
\item Adicione a água fervendo com e o vinagre. Cuidado que espirra, proteja-se com uma tampa.  Mexa imediatamente até dissolver.
\item Deixe em ferver em fogo baixo até chegar ao ponto de fio. Verifique o ponto tomando um pouco com a colher e despejando. Se a calda descer em fio ao invés de parecer um líquido fluido estará no ponto.
\item Espere esfriar para usar.
\end{enumerate}
\textbf{Pudim}
\begin{enumerate}
\item Bata todos os ingredientes no liquidificador.
\item Passe calda por todo o interior de uma forma de pudim.
\item Despeje a mistura batida na forma.
\item Leve a forma a banho maria por até que o pudim esteja cozido e sólido (30-40 min no fogão a fogo baixo).
\item Se estiver usando uma panela de banho maria tire a água e deixe no fogo por mais 30 a 60 segundos, para caramelizar ainda mais a calda.
\item desligue o fogo e deixe a forma esfriar completamente. Coloque na geladeira até que fique gelado.
\item Passe uma faca fina pelas bordas do pudim para separá-lo da forma. Desenforme em uma prato grande.
\end{enumerate}
\subsection{Macetes}
\begin{itemize}
\item A receita de calda é suficiente para vários pudins. Você pode fazê-la com antencedência e guardar em um vidro na geladeira.
\item Você pode aumentar a quantidade de calda no pudim se gostar. Mas certifique-se que todo o interior da forma esteja revestido de calda antes de jogar a mistura.
\item Há formas de banho-maria para fogão, como a que mostramos nas figuras. Com elas gasta-se menos gás e o pudim fica pronto mais rápido. 
\item O maior segredo é o ponto de cozimento: deve ser o mínimo para que não desmonte, mas também não pode cozinhar demais ou fica borrachento. Com alguma experimentação você vai acertar.
\end{itemize}
